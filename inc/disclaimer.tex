\subsection{Disclaimer}
Penetrační test je obvykle popisován jako přesná a kompletní simulace útoku na danou službu či aplikaci. Ačkoliv penetrační test a reálný útok mají mnoho společného, například znalosti testera a útočníka, používané nástroje a další, tak zde existuje i několik podstatných rozdílů, které je potřeba brát v úvahu. Jedná se především o omezení penetračního testu penězi či časem.

V případě reálného útoku může útočník plánovat útok i několik měsíců předem. Může si tedy dlouhodobě shromažďovat informace potřebné k útoku. V případě penetračního testu si takovýto luxus dovolit nelze, jelikož takový penetrační test by byl finančně neúnosný a zároveň by nepřinesl požadovaný výsledek v rozumném čase pro případná protiopatření.

Z tohoto důvodu je občas potřeba při penetračním testu určitá součinnost od testovaného subjektu, aby v čase vymezeném pro penetrační test bylo možno otestovat systémy co nejvíce.
