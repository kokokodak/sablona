\section{Classification of vulnerabilities}
Classification of identified vulnerabilities is described in this chapter. Each vulnerability is assigned severity according to seriousness of consequences from exploitation of such vulnerability.

\subsection{According to severity}

\begin{itemize}
\item Critical \nobreak\smallskip\protect\includegraphics[width=7pt]{critical-small} \\
Serious vulnerability that has direct consequences for the security of tested system.

\item High \nobreak\smallskip\protect\includegraphics[width=7pt]{high-small} \\
Vulnerability that can have direct consequences for the security of tested system in case of skilled or motivated attacker.

\item Medium \nobreak\smallskip\protect\includegraphics[width=7pt]{medium-small} \\
Vulnerability that isn't serious alone, but in combination with other vulnerability/vulnerabilities can have direct consequences for the security of tested system.

\item Low \nobreak\smallskip\protect\includegraphics[width=7pt]{low-small} \\
Finding that has low impact on security of tested system, it is usually part of best practices.

\item Informative \nobreak\smallskip\protect\includegraphics[width=7pt]{info-small} \\
Finding is of informative severity and isn't security vulnerability.

\end{itemize}

