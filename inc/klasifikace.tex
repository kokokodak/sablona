\section{Klasifikace zranitelností}
V této kapitole je popsána klasifikace jednotlivých nalezených zranitelností. Každé zranitelnosti je přiřazena míra závažnosti podle dopadů, které by zneužití dané chyby mělo na testovaný systém.

\subsection{Podle míry závažnosti}
Závažnost jednotlivých nálezů vychází z jejich dopadů na celkové zabezpečení systému.
Během hodnocení závažnosti jednotlivých nálezů je zohledněno případné využití společně s ostatními nalezenými zranitelnostmi.


\begin{itemize}
\item Kritická \nobreak\smallskip\protect\includegraphics[width=7pt]{critical-small} \\
Závažná zranitelnost, která má přímé dopady na celkovou bezpečnost testovaného systému.

\item Vysoká \nobreak\smallskip\protect\includegraphics[width=7pt]{high-small} \\
Zranitelnost, která může mít přímé dopady na bezpečnost testovaného systému v případě schopného či motivovaného útočníka.

\item Střední \nobreak\smallskip\protect\includegraphics[width=7pt]{medium-small} \\
Zranitelnost, která sama o sobě nemá dopad na celkovou bezpečnost testovaného systému, nicméně společně s dalšími zranitelnostmi může představovat určité riziko.

\item Nízká \nobreak\smallskip\protect\includegraphics[width=7pt]{low-small} \\
Nález, který má nízký dopad na celkovou bezpečnost testovaného systému, jedná se většinou spíše o best practices.

\item Informativní \nobreak\smallskip\protect\includegraphics[width=7pt]{info-small} \\
Nález, který je čistě informativní povahy a nepředstavuje bezpečnostní problém.

\end{itemize}

