\subsection{Disclaimer}
Vulnerability scan slouží k prvotnímu zběžnému posouzení bezpečnosti scanovaného serveru. V případě vulnerability scanu nedochází, narozdíl od penetračního testu, k ověřování nalezených zranitelností a ručnímu hledání zranitelností. Z tohoto důvodu je pravděpodobné, že ve výsledcích z vulnerability scanu se může nacházet určité množství tzv. false positive, což jsou nálezy, které se ve skutečnosti na serveru nenacházejí. Zároveň je velice pravděpodobné, že při vulnerability scanu nebyly nalezeny všechny zranitelnosti, které se na serveru nacházejí.

