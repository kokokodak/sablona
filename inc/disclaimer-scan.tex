\subsection{Disclaimer}
Vulnerability scan slouží k prvotnímu zběžnému posouzení bezpečnosti webové aplikace. V případě vulnerability scanu nedochází, narozdíl od penetračního testu, k ověřování nalezených zranitelností, testování autentizované části webové aplikace a ručnímu hledání zranitelností. Z tohoto důvodu je pravděpodobné, že ve výsledcích z vulnerability scanu se může nacházet určité množství tzv. false positive, což jsou nálezy, které se ve skutečnosti ve webové aplikaci nenacházejí. Zároveň je velice pravděpodobné, že při vulnerability scanu nebyly nalezeny všechny zranitelnosti, které se v aplikaci nacházejí.

